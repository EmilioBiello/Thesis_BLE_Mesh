\chapter{Conclusione}
\label{ch:conclusione}
Nel seguente elaborato di tesi è stata descritta la progettazione e la valutazione sperimentale di reti di monitoraggio mesh indoor basate su tecnologia Bluetooth Mesh.\\

\noindent Durante la prima fase è stato necessario uno studio preliminare degli aspetti tecnologici relativi alla situazione attuale in merito al mondo IoT, in modo da comprendere il progresso tecnologico che sta caratterizzando le reti di sensori e più nello specifico il nuovo standard rilasciato da Bluetooth SIG per consentire la comunicazione \textit{many-to-many} impiegando una radio Bluetooth. Tale standard opera al di sopra dello stack protocollare Bluetooth Low Energy, quindi, prima di procedere con la realizzazione di una rete mesh è stato necessario comprendere a pieno il protocollo sottostante.\\
Comprese le caratteristiche dei due standard forniti da Bluetooth SIG, si è passati alla comprensione del framework rilasciato da Espressif per configurare le proprie board e solo allora è stato possibile approfondire nel dettaglio l’impiego e l'utilizzo delle librerie messe a disposizione per realizzazione di una rete mesh di microcontrollori avvalendosi di dispositivi ESP32. La difficoltà maggiore ha riguardato l'utilizzo della libreria messa a disposizione da Espressif poiché non ancora in versione definitiva, quindi sempre in continua evoluzione. Evoluzione necessaria poiché il supporto a tale standard è piuttosto recente (secondo la documentazione, l'inizio di un progetto dedicato a Bluetooth Mesh è avvenuto a partire da gennaio 2019).\\
La scelta di focalizzare il lavoro sulle tecnologie Bluetooth e Wi-Fi ha riguardato principalmente la loro diffusione, ovvero oggigiorno sono presenti in qualsiasi tipologia di apparato di utilizzo quotidiano. Quindi utilizzando l'hardware comunemente a disposizione di chiunque è possibile creare un rete mesh, adoperando un software specifico sufficiente a garantire tale modalità di funzionamento.\\
Tra le due si è preferito il maggior impiego della tecnologia Bluetooth, poiché risulta essere meno dispendiosa energeticamente. In merito alla tecnologia Wi-Fi, purtroppo non è stato possibile utilizzarla in modalità Mesh, poiché i dispositivi in possesso, per questioni di memoria e di scarsa ottimizzazione delle librerie messe a disposizione dalla casa costruttrice, non sono in grado di eseguire i due stack contemporaneamente. Per fronteggiare tale situazione è stata utilizzata la tecnologia Bluetooth in modalità Mesh e quella Wi-Fi secondo il metodo convenzionale, ovvero con la presenza di un Access Point per garantire la comunicazione wireless tra i nodi della rete.\\
I risultati ottenuti dalle simulazioni eseguite utilizzando in combinazione lo standard Bluetooth Mesh e lo standard Wi-Fi hanno largamente soddisfatto quelle che erano le aspettative del seguente progetto di tesi.\\
In base ai risultati riscontrati dai test, la tecnologia Bluetooth Low Energy risulta essere efficiente dal punto di vista energetico ed in grado di garantire un'ottima copertura in un ambiente indoor piuttosto ampio andando ad utilizzare lo standard Bluetooth Mesh, oltre ad un basso costo delle componentistiche. L'unica nota dolente riguarda le sue prestazioni con un carico di lavoro piuttosto elevato, in cui si verificano elevate perdite di pacchetti. A tal proposito, utilizzando la tecnologia Wi-Fi è stato possibile alternare l'invio dei pacchetti tra le due tecnologie, garantendo così maggior affidabilità nella comunicazione.\\

% \noindent Il lavoro svolto si è dimostrato impegnativo in termini di impegno richiesto ma anche in termini di tempo, in quanto si è trattato di un progetto con tanti requisiti da soddisfare lato implementativo, il che hanno portato ad una lunga fase di validazione e valutazione. Tale lavoro e tale impegno sono stati ripagati con il soddisfacimento degli obiettivi preposti, ovvero andando ad eseguire su un medesimo nodo due stack protocollari differenti e alternare uno o l'altro in fase di comunicazione in modo da mantenere alte le prestazioni della rete, soprattutto nei casi in cui, utilizzando un unico stack, risulterebbe congestionata.


\section{Sviluppi Futuri}
Questo lavoro di tesi vuole essere un punto d'inizio per una serie di possibili sviluppi futuri che andrebbero sicuramente a migliorare molti aspetti, legati al comportamento del nodo in merito alle tecnologie di comunicazione impiegate, che per motivi di tempo non sono stati trattati.\\
Uno dei possibili sviluppi futuri può riguardare l'introduzione di intelligenza artificiale all'interno della rete, attraverso il paradigma di Machine Learning. Andando ad istruire il nodo attraverso tale tecnica, sarà in grado di acquisire conoscenza relativa all'ambiente circostanze, al carico di lavoro dell'intera rete e di come esso sia ripartito tra le due tecnologie, nonché le capacità di far fronte a tale situazione, le interferenze presenti in tale ambiente e le conseguenti perdite di pacchetti che si andranno a verificare.
Con tale base di conoscenza sarà in grado di effettuare delle scelte riguardanti la tecnologia da utilizzare per ogni singola comunicazione.
La scelta dovrà comunque tenere in considerazione i consumi energetici, in fase di trasmissione, della tecnologia utilizzata. 
Rispettando tale specifiche dovrà cercare allo stesso tempo di mantenere alte le prestazioni della rete, ovvero cercando di garantire sempre un Packet Delivery Ratio prossimo al valore 1.\\
Un altro progetto potrebbe prevedere il coinvolgimento di un'altra tipologia di standard da accostare a Bluetooth Mesh, con l'intendo di minimizzare i consumi energetici pur mantenendo alte le prestazioni della rete. Ad esempio, si potrebbe far ricorso ad una tecnologia appartenente alla categoria Wireless Personal Area Network, scegliendo tra ZigBee e 6LowPAN. Dopodiché eseguire le medesime sperimentazioni e confrontare i risultati acquisiti, al fine di poter decretare quale combinazione risulta essere meno dispendiosa e allo stesso tempo performante.
Un ulteriore progetto potrebbe inglobare il seguente lavoro ed eseguire ulteriori valutazioni prestazionali che, per questioni di tempo o di componenti non è stato possibile eseguire. Questo sviluppo potrebbe coinvolgere un misuratore di consumi energetici, in modo da misurare costantemente i consumi effettivi e applicare delle policy per regolamentarli. Un altro aspetto potrebbe riguardare la realizzazione di una rete più ampia, impiegando dei nodi aventi la feature di ``Friend'' attiva, in modo da inglobare anche i Low Power Node. Dopodiché ripetere le valutazioni prestazionali svolte o aggiungervene di nuove.