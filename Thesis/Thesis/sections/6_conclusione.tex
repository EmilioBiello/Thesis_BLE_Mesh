\chapter{Conclusione}
\label{ch:conclusione}
Nel seguente progetto di tesi è stata descritta la progettazione e la valutazione sperimentale di reti di monitoraggio mesh indoor basate su tecnologia Bluetooth Mesh.\\
Lo studio inizia con una panoramica su quello che è lo stato dell’arte, per renderci conto del progresso tecnologico che sta caratterizzando le reti di sensori, focalizzando la nostra attenzione su una tecnologia piuttosto innovativa, fornita dallo Bluetooth SIG.\\
Assodati i tratti salienti che caratterizzano il protocollo di comunicazione, si è cercato di progettare le basi per un sistema in grado alternare la tecnologia usata per la comunicazione in base al comportamento percepito della rete di appartenenza.  
I risultati ottenuti dalle simulazioni eseguite utilizzando in combinazione lo standard Bluetooth Mesh e lo standard Wi-Fi hanno largamente soddisfatto quelle che erano le aspettative del seguente progetto di tesi.\\
In base ai risultati riscontrati dai test, la tecnologia Bluetooth Low Energy risulta essere efficiente dal punto di vista energetico ed in grado di garantire un ottima copertura in un ambiente indoor piuttosto ampio andando ad utilizzare lo standard Bluetooth Mesh, oltre ad un basso costo componentistiche. L'unica nota dolente riguarda le sue prestazioni con un carico di lavoro piuttosto elevato, in cui si verificano innumerevoli perdite di pacchetti. A tal proposito, utilizzando la tecnologia Wi-Fi è stato possibile alternare l'invio dei pacchetti tra le due tecnologie, garantendo così affidabilità della connessione.\\

\section{Sviluppi Futuri}
A partire dal seguente progetto di tesi, sono inoltre possibili future implementazioni per quanto riguarda le funzionalità del nodo e le sue modalità di utilizzo.\\
Uno dei possibili sviluppi futuri può riguardare l'introduzione di intelligenza artificiale all'interno della rete, attraverso il paradigma del Machine Learning. Istruendo il nodo attraverso tale tecnica, sarà in grado di fare delle scelte riguardanti la tecnologia da utilizzare per la singola comunicazione tenendo in considerazione quanto appreso finora. Le scelte saranno influenzate dalla situazione corrente della rete e dai consumi della tecnologia in questione.
