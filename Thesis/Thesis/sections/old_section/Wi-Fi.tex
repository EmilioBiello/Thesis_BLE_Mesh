\section{Wi-Fi}
% https://www.wi-fi.org/discover-wi-fi --> Wi-Fi Alliance
% https://it.wikipedia.org/wiki/IEEE_802.11
Esistono diversi standard per consentire di utilizzare la comunicazione wireless per raccogliere informazioni o eseguire determinate attività di controllo. Tra di essi è possibile individuare anche la famiglia dello standard IEEE 802.11, comunemente utilizzato per fornire accesso ad internet all'interno di un'area limitata (WLAN).\\
Il Wi-Fi è una tecnologia wireless che consente a dispositivi come laptop, smartphone, Tv, etc di connettersi ad alta velocità ad Internet senza la necessità di una connessione cablata ma attraverso delle onde radio appartenenti alla banda Industrial, Scientific e Medical (ISM) poiché utilizzabile liberamente senza l'impiego di una licenza.\\
Dato che l'\textit{Institute of Electrical and Electronic Engineers} (IEEE) pubblica solo un insieme di specifiche, ma non prevede alcun test per riconoscere se un prodotto rispetti le suddette specifiche, è stata fondata, nel 1999, la Wi-Fi Alliance. L'associazione racchiude diversi costruttori hardware (Cisco, Netgear, Nokia, Intel, Broadcom, Philips, ASUS, ecc) ed è nata con l'obiettivo di certificare l'interoperabilità dei prodotti e diffondere l'utilizzo delle reti Wi-Fi.\\

% https://en.wikipedia.org/wiki/IEEE_802.11#Protocol
% https://www.netspotapp.com/explaining-wifi-standards.html
\noindent La famiglia 802.11 è costituita da quattro protocolli dedicati semplicemente alla trasmissione delle informazioni (a, b, g, n), mentre gli aspetti legati alla sicurezza sono definiti in standard a parte (802.11i).\\
La prima versione dello standard WiFi è stata rilasciata nel 1997 garantendo una velocità di trasmissione compresa tra 1 e 2 Mbps e adottando come tecnica di trasmissione dati sia le onde radio nella frequenza 2.4 GHz sia i raggi infrarossi, quest'ultima eliminata dalle versioni successive a causa del suo scarso successo, ovvero per la sua bassa capacità di oltrepassare gli ostacoli. La trasmissione attraverso le microonde (banda ISM) prevedeva l'uso di due tecniche: la Frequency Hopping Spread Spectrum (FHSS) e la Direct Sequence Spread Spectrum (DSSS).\\
Tale standard è stato rapidamente soppiantato dall'introduzione, nel 1999, delle specifiche 802.11a e 802.11b in grado di fornire una velocità di trasmissione nettamente maggiore rispetto al precedente (rispettivamente 54 Mbps e 11 Mbps). La specifica 802.11a, chiamata anche WiFi 2,ha introdotto l'utilizzo della banda 5 Ghz e l'utilizzo della tecnica OFDM a livello di collegamento per consentire la trasmissione dei dati, mentre 802.11b, denominato anche WiFi 1, risulta essere un'estensione diretta dello standard originale.\\
Nel giugno del 2003 è stato rilasciato il terzo standard (WiFi 3), 802.11g, operante sempre nella banda 2.4 GHz (come 802.11b) ma utilizza lo schema di trasmissione basato su OFDM (come 802.11a) garantendo un tasso trasmissivo massimo di 54 Mbps.\\
Nel ottobre del 2009 è stato rilasciato lo standard 802.11n, in grado di migliorare gli standard precedenti ed etichettato come WiFi 4 dalla Wi-Fi Alliance. Lo standrdard ha introdotto le antenne multiple-input multiple-output (MIMO), è in grado di funzionare su entrambe le frequenze con una velocità dati che varia dai 54 Mbps ai 600 Mbps.\\
Oltre alle versioni fin qui menzionate, la famiglia 802.11 comprende ulteriori standard (c, d, e, f, h, ...), il cui compito riguarda l'estensione dei servizi di base.\\
Tornando al contesto IoT, le versioni discusse sinora sono utilizzate in tale contesto purché non risultano ottimali per il dispendioso consumo energetico. A tal proposito sono stati introdotti due standard specificamente per l'IoT: il WiFi HaLow (802.11 ah) introdotto nel 2017, in grado di operare su banda, sempre esente da licenza, inferiore a 1 GHz; e il HEW (802.11 ax) commercializzato come WiFi 6 sarà il successore dello standard 802.11ac e andrà ad aumentare l'efficienza delle reti WLAN.\\

% Developing ZigBee Deployment Guideline Under WiFi Interference for Smart Grid Applications
% http://edu.majornet.it/tutto-sul-wi-fi
% https://www.electronics-notes.com/articles/connectivity/wifi-ieee-802-11/what-is-wifi.php
\noindent Lo standard 802.11 definisce il livello PHY e MAC per il WiFi per implementare una Wireless Local Area Network. 
Le reti 802.11 a 2.4 GHz (banda compresa tra 2.4 GHz e 2.485 GHz) dividono lo spettro in 14 canali con ampiezze variabili in base alla topologia del protocollo utilizzato. La disponibilità dei canali è regolata dalle nazioni in base a come è assegnato lo spettro radio ai vari servizi. Al momento in Europa è consentito l'uso dei canali da 1 a 13, mentre per gran parte dei paese americani l'uso è ristretto ai canali da 1 a 11. 
Nonostante quanto detto prima in alcune circostanze, anche in Europa, per via delle schede di rete presenti nei dispositivi utilizzati, si è costretti ad utilizzare solo 11 canali. Questo è dovuto alla provenienza delle schede di rete, per quelle fabbricate negli Stati Uniti è vietato adoperare i canali 12 e 13 poiché tali frequenza risultano molto vicine a quelle utilizzate da un'azienda privata fornitrice di un servizio di telefonia satellitare, basato su satelliti a bassa orbita, sull'intero territorio americano.\\
Le reti 802.11 a 5 GHz dividono lo spettro in 30 canali con ampiezza di banda a partire da 20 MHz, di cui 23 non sovrapposti, fino ad un massimo di 160 MHz. Tale banda non viene adoperata in un contesto di IoT poiché garantisce un raggio di copertura inferiore rispetto alla banda 2.4 GHz ed una minor capacità di superare gli ostacoli.\\
I canali della banda 2.4 GHz sono parzialmente sovrapposti tra loro, quindi esiste forte interferenza tra due canali consecutivi il che comporta una perdita di informazioni. Due canali non si sovrappongono solo se risultano separati da almeno quattro canali. Difatti, nel medesimo luogo fisico possono essere occupati soltanto 3 canali contemporaneamente senza avere interferenze. La principale terna di canali non sovrapposti utilizzata nella maggior parte dei continenti corrisponde a 1, 6, 11 con un'ampiezza di banda di 22 Mhz (802.11b), mentre nel caso della modulazione OFDM (802.11g), l'ampiezza è di 20 MHz e la sequenza corrisponde ad 1, 5, 9, 13.\\
IEEE 802.11b ha una velocità di trasmissione massima di 11 Mbps ed utilizza il meccanismo \textit{CSMA-CA} per l'accesso al mezzo trasmissivo, cercando di evitare, quando possibile, le collisioni. 
La modulazione del segnale avviene attraverso la tecnica \textit{Direct-Sequence Spread Spectrum} (DSSS) ed utilizza un ampiezza di banda a 22 MHz. 
L'intera famiglia di protocolli 802.11 prevede anche di ridurre la frequenza trasmissiva per raggiungere distanza più consistenti e possono funzionare sia in modalità ``infrastruttura'' sia in modalità ``ah hoc''. La modalità infrastruttura prevede la presenza di un Router Ethernet a cui risulta collegato un Access Point, o in alcuni casi risulta inglobato direttamente all'interno del router. La modalità ad hoc, non prevede nessuna presenza di dispositivi intermedi, ovvero una rete priva di controllo centrale e di connessioni con il mondo esterno costituita esclusivamente da dispositivi wireless e solitamente utilizzata quando si vogliono condividere informazioni localmente tra un gruppo di persone.\\

% LIbro Reti di calcolatori
\noindent Lo standard 802.11 prevede la creazione di un \textit{Service Set}, ovvero un gruppo di dispositivi wireless connesso ad un rete attraverso il suo SSID (Service Set Identifier), nome univoco che identifica una rete WiFi, e solitamente a seguito di una procedura di autenticazione. Un insieme di dispositivi e la stazione base centrale, detta anche Access Point (AP), costituiscono il \textit{Basic Service Set} (BSS). Una rete può essere composta da diversi access point, che fungono da sorgente di segnale, e da uno o più client. L'insieme di BSS collegate tra loro a livello MAC, costituisce una \textit{Extended Service Set} (ESS).\\
Le stazioni wireless prima di poter inviare o ricevere un pacchetto devono associarsi ad un AP, identificabile tramite il proprio SSID. Una volta che la stazione subentra all'interno della rete wifi può iniziare a trasmettere e ricevere frame dati da e verso l'AP, adoperando il meccanismo CSMA-CA per l'accesso al mezzo. Vale a dire che ciascuna stazione ascolta il canale prima di trasmettere e si astiene dal farlo se rileva lo trasmissione di un'altra stazione. Al momento che il canale risulta libero, procede con la trasmissione del dato e procede finché l'intero frame non risulta inviato. Siccome la comunicazione wireless comporta perdita di pacchetti, per via delle interferenze, è stato introdotto lo schema di ``avvenuta ricezione/ritrasmissione'' a livello di collegamento \cite{kurose2013reti}.\\
Al fine di garantire che la rete locale rimanga sicura è richiesta una procedura di autenticazione, la quale inizialmente prevedeva l'utilizzo di una Wired Equivalent Privacy (WEP) e in seguito sostituita, a causa di gravi carenze, da Wi-Fi Protected Access (WPA) costituita da 3 livelli di sicurezza (WPA, WPA2 e WPA3).\\

% https://www.leverege.com/iot-ebook/iot-wifi
\noindent La tecnologia Wi-Fi è in grado di inviare elevate quantità di dati a discapito di un dispendioso consumo energetico e di una portata ridotta. Questa tecnologia può essere utile per le applicazioni IoT che non devono preoccuparsi del consumo energetico, poiché alimentate a corrente, e che devono inviare molti dati, come video, e non richiedono un raggio di comunicazione elevato. Un esempio potrebbe essere un sistema di sicurezza domestica.