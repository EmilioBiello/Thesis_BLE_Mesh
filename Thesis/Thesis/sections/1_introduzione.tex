\chapter*{Introduzione}
\label{ch:introduzione}

\rhead[\fancyplain{}{\bfseries INTRODUZIONE}]{\fancyplain{}{\bfseries\thepage}}
\lhead[\fancyplain{}{\bfseries\thepage}]{\fancyplain{}{\bfseries INTRODUZIONE}}
\addcontentsline{toc}{chapter}{Introduzione}

L'Internet of Things (IoT) è un paradigma innovativo che sottende alla capacità di estendere la rete Internet connettendo oggetti di uso quotidiano.
% influenza la nostra vita dal modo di reagire al modo di comportarci. Difatti consente la creazione di una gigantesca network composta da dispositivi connessi, ovvero qualsiasi tipologia di oggetto di uso quotidiano dotato di una connessione Internet. 
Suddetti dispositivi spesso raccolgono e condividono dati e informazioni relativi al loro utilizzo e dell'ambiente in cui essi operano, avvalendosi degli svariati sensori embedded di cui ogni \textit{smart object} risulta dotato, al fine di aiutare gli utenti a migliorare l'operatività, risparmiando tempo prezioso e denaro. \\
Tale fenomeno, seppure di recente diffusione, ha raggiunto numeri notevoli. Basti pensare che per fine 2020 sono stimati 31 miliardi di dispositivi connessi e poiché questo andamento continuerà a crescere, le previsioni per il 2025 indicano il raggiungimento di ben oltre 75 miliardi di dispositivi connessi, con un impatto sull'economia dell'ordine di 1.6 trilioni di dollari U.S. \cite{statista2020iot, statista2016iot}.\\

\noindent La rapida diffusione e l'impiego così elevato di dispositivi connessi, di varia natura, ha portato alla progettazione di architetture di rete e protocolli al fine di garantire la comunicazione tra i nodi, nonché alla definizione delle Wireless Personal Area Network (WPAN) e Wireless Sensor Network (WSN).\\
Una WPAN corrisponde ad un tipo specifico di PAN (Personal Area Network) in cui tutte le connessioni sono wireless. Prevede la connettività tra apparecchi elettronici, dispositivi di comunicazione e computer tipicamente su distanze dell’ordine dei 10 metri, standardizzate dall’IEEE nell’ambito del gruppo di lavoro 802.15. All'interno delle Wireless PAN è possibile considerare la categoria delle WSN, ovvero una particolare tipologia di reti wireless ideata per dispositivi a basso costo, di limitate dimensioni e modeste prestazioni. Tali dispositivi, distribuiti in uno spazio senza la necessità di un'infrastruttura e dotati di sensori, cooperano tra loro alla scopo di monitorare l'ambiente circostante.\\
Per garantire la comunicazione tre i nodi appartenenti ad una rete sono stati definiti diversi standard, tra cui gli standard ZigBee e 6LowPAN definiti sul livello fisico e di collegamento espresso dallo standard 802.15.4 e lo standard Bluetooth Mesh, rilasciato nel luglio del 2017 da Bluetooth SIG. Con la definizione del suddetto standard si è avuta la possibilità di utilizzare la radio Bluetooth, una tecnologia ampiamente diffusa nei dispositivi di uso quotidiano, per garantire una comunicazione \textit{many-to-many} e quindi consentire la creazione di reti di dispositivi su larga scala.
% poter essere utilizzata per far dialogare i nodi appartenenti ad una rete. 
Tale standard opera al di sopra dello stack protocollare definito da Bluetooth Low Energy, sfruttando così i vantaggi garantiti da quest'ultimo. \MakeUppercase{è} ideale per sistemi di controllo, di monitoraggio e automazione in cui decine, centinaia o migliaia di dispositivi devono comunicare in modo affidabile e sicuro.\\

\noindent Lo scopo del lavoro oggetto di questa tesi riguarda la progettazione e la valutazione di una rete mesh in ambiente indoor impiegando la tecnologia Bluetooth Mesh. Inizialmente vengono illustrate tutte le peculiarità che costituiscono lo standard di riferimento, per poi passare alla progettazione e all'implementazione di una rete mesh costituita da nodi in grado di comunicare attraverso questa tecnologia.\\ 
Nell'implementazione dei nodi che andranno a costituire la rete si è operato procedendo per step: un primo step in cui si è scelto di utilizzare solo la tecnologia Bluetooth e valutare il comportamento della rete così realizzata; un secondo step in cui alla tecnologia Bluetooth è stata affiancata una tecnologia di supporto, in modo da far fronte alle problematiche ipotizzate e poi emerse con l'aumentare del carico di lavoro.\\ 
Per operare con due tecnologie su un medesimo nodo, è stato necessario implementare un algoritmo in grado di combinare il loro impiego. Tale combinazione avviene analizzando costantemente le condizioni della rete, ovvero comprendendo in tempo reale un'eventuale congestione che potrebbe causare perdite di pacchetti o ritardi di accodamento. Qualora venga individuato un tale evento si procederà con la relativa ritrasmissione impiegando la tecnologia di supporto.
L'algoritmo così implementato prevede due fasi, una prima fase denominata ``sleep mode' in cui verrà impiegata la sola tecnologia Bluetooth per la trasmissione dei dati e terminerà nel momento in cui la rete giungerà in uno stato definito ``pieno regime''. 
Una volta raggiunta questa condizione, si passerà alla fase due in cui si continuerà a trasmettere i dati utilizzando la tecnologia Bluetooth, però allo stesso tempo, verrà analizzato il traffico presente nella rete mesh al fine di identificare dei pacchetti persi, i quali verranno trasmessi nuovamente, questa volta però, tramite la tecnologia Wi-Fi. \\
La trasmissione dei dati avverrà, per entrambe le tecnologie, utilizzando la medesima frequenza trasmissiva, legata alla tipologia di test che verrà eseguito.\\
L'algoritmo analizzando costantemente lo stato della rete, provvederà ad aggiornare un valore di soglia, impiegato per definire quando considerare un pacchetto ``perso''. Difatti una volta che un pacchetto verrà immesso nella rete, l'algoritmo procederà a calcolare il tempo trascorso dal suo invio e qualora esso risultasse maggiore della soglia verrà definito come perso e innescherà un meccanismo di ritrasmissione.\\ 
Il fulcro di tale lavoro riguarda proprio la definizione e l'implementazione dell'algoritmo che consente la combinazione delle tecnologie BLE e Wi-Fi su un medesimo nodo, alternandole come descritto poc'anzi, al fine di regolamentare la comunicazione all'interno della rete mesh in modo efficace, evitando il più possibile la sua congestione e quindi la relativa perdita di pacchetti. \\ 
% Tale tecnologia consente svariate personalizzazioni, ma anche alcune limitazioni relative alla comunicazione, presupposte e individuate nel momento in cui si è andato ad aumentare il carico di lavoro. A tal proposito è stata definita una metodologia di comunicazione personalizzata ed è stata affiancata una tecnologia di supporto al fine di garantire delle prestazioni elevate anche in una simile circostanza. Per garantire ciò è stato necessario definire un algoritmo in grado di analizzare costantemente lo stato della rete e veicolare la comunicazione tra le due tecnologie.\\
Conclusa la fase implementativa dei nodi in rispetto delle specifiche definite, si è passati alla creazione e valutazione della rete, definendo dei testbed che permettessero di analizzare il comportamento sia al variare della distanza sia al variare del carico di lavoro. \\
Infine, i risultati ottenuti sono stati confrontati tra loro e hanno dimostrato quanto supposto inizialmente, ovvero, la combinazione dei due stack protocollari sul medesimo nodo gestiti mediante l'algoritmo dinamico (sopra descritto), comporta notevoli vantaggi in termini prestazionali, misurati attraverso la metrica Packet Delivery Ratio.\\
% in grado di consentire alla rete di far fronte ad una situazione di congestione alternando le due tecnologie, cosa che non si sarebbe mai verificata utilizzando esclusivamente lo standard Bluetooth Mesh.\\

\noindent La tesi si articola in cinque capitoli. 
Nella prima parte (capitolo 1 e capitolo 2) viene presentato lo stato dell’arte. Nel capitolo 1, in particolar modo viene presentata una panoramica del mondo dell'Internet of Things (IoT) e del paradigma Machine-to-Machine (M2M), già esistente al momento della diffusione dell'IoT. Si passa poi alla descrizione degli standard che regolamentano Wireless Personal Area Network (WPAN). Per prima cosa viene descritto lo standard alla base del seguente progetto di tesi, ovvero Bluetooth Low Energy (BLE) per poi passare ad una breve descrizione del protocollo 802.15.4 fino ad addentrarci su due tecnologie che adottano il livello fisico e di collegamento definito dal suddetto standard, ovvero le tecnologie ZigBee e 6LowPAN.
Il capitolo 2 è interamente dedicato all'elemento cardine del seguente elaborato, ovvero lo standard  Bluetooth Mesh il quale ha reso possibile l'impiego di una comunicazione \textit{many-to-many} attraverso la tecnologia BLE (fino al luglio 2017, in grado di garantire solo una comunicazione \textit{one-to-one} o \textit{many-to-one}).
La seconda parte dell'elaborato (capitolo 3, 4 e 5) riguarda in maniera specifica il progetto sviluppato.
Nel capitolo 3 viene presentata l'idea alla base, andando a descrivere le scelte progettuali. Nel capitolo 4 vengono illustrate le scelte implementative, nonché le tecnologie utilizzate e i moduli implementati. Infine il capitolo 5 concluderà il lavoro illustrando lo scenario applicativo, le valutazioni sperimentali e i risultati ottenuti.