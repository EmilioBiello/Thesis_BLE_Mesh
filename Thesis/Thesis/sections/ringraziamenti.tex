\chapter*{Ringraziamenti}
\thispagestyle{empty}
\pagestyle{empty}
Eccomi giunto alla fine di questa tesi e di questi splendidi anni universitari, nei quali credo di essere maturato come professionista in quella mia grande passione che è l'Informatica, ma anche e soprattutto come persona. Sono tante le conoscenze che ho fatto durante questo percorso, le amicizie che ho coltivato, i rapporti che ho stretto. Vorrei dedicare queste ultime pagine per ringraziare tutte le persone che hanno sempre creduto in me e che mi hanno sempre sostenuto sia nei momenti di difficoltà sia in quelli felici e spensierati. Vorrei che questi ringraziamenti siano un punto d'arrivo (per la carriera universitaria), ma anche un punto d'inizio, perché credo che non si finisca mai di crescere e spero di poter raggiungere nuovi traguardi importanti nella mia vita con tutti voi ancora al mio fianco.\\

\noindent In primis vorrei ringraziare il mio Relatore, Prof. Marco Di Felice, non solo per la fiducia accordatami accettando il ruolo di Relatore e affidandomi un progetto così innovativo, ma soprattutto per avermi spronato nel fare sempre di più e non accontentarsi mai dei traguardi raggiunti, ma spingersi sempre oltre, per poter raggiungerne di nuovi.
Inoltre lo ringrazio per la professionalità, la chiarezza e la pazienza con cui mi ha seguito durante la stesura del mio elaborato, ma anche per tutto ciò che mi ha insegnato durante il suo corso, suscitando in me un forte interesse nell'ambito dell'Internet of Things.\\ 
Il mio Correlatore, il Dottor Angelo Trotta per la disponibilità e la pazienza dimostrata ogni qual volta mi recavo da lui per avere maggiori delucidazioni sull'argomento, per i preziosi consigli divulgati e soprattutto per avermi trasmesso una forte passione su questa tematica.\\

\noindent Un doveroso ringraziamento va ovviamente alla mia famiglia, senza la quale non avrei mai neppur cominciato questa carriera. Mi è sempre stata accanto e non ha fatto mai mancare il suo sostegno e il suo aiuto durante tutti questi anni. Grazie perché senza di voi non sarei mai arrivato fino in fondo a questo difficile, lungo e tortuoso cammino. Senza di voi non sarei mai diventato quello che sono e non avrei potuto coronare i miei molteplici sogni.\\
Ai miei genitori, che sono il mio punto di riferimento e che mi hanno sostenuto sia economicamente che emotivamente. Questa tesi la dedico a voi che siete la mia famiglia, il mio più grande sostegno e la mia guida.\\

\noindent Vorrei ringraziare Mario, un amico, un compagno di mille avventure, un ragazzo con cui ho condiviso attività di qualsiasi genere partendo dallo sport fino al sociale. So che per qualsiasi cosa potrò contare sempre su di te.\\

\noindent Un ringraziamento speciale a Simone e Aldo, che spinti dalle stesse passioni abbiamo condiviso tantissime esperienze assieme fino a diventare anche coinquilini, e che coinquilini. A seguito delle differenti scelte e delle carriere intraprese, ci siamo dovuti allontanare, ma la nostra amicizia non ha paura di una tale distanza, anche perché, le amicizie, quelle vere sono in grado di resistere al tempo, alle distanze e al silenzio. So che possiamo sempre contare gli uni sugli altri. \\

\noindent Vorrei ringraziare anche Angelo, che è stato più che un coinquilino. Nel corso di questi anni si è rivelato un vero e sopratutto simpaticissimo amico, sempre pronto ad ascoltarti e ad aiutarti. Il legame che si è creato del corso del tempo resterà forte e duraturo nonostante i numerosi chilometri che ci separano. Anche se ora, a seguito delle tue scelte, risultano essere molto di meno.\\

\noindent Un altro ringraziamento speciale voglio dedicarlo a Cristina, che nonostante la lontananza è sempre presente, pronta a spronarmi nelle mie scelte e a dispensare preziosi consigli. Vorrei ringraziare anche Annamaria per le innumerevoli serate trascorse assieme, talvolta quasi a sfiorar l'alba, nonostante il mattino seguente dovevi svegliarti presto per andare al lavoro. Inoltre, voglio ringraziane tutti gli amici di Casacalenda, i quali hanno tutti avuto un peso nel conseguimento di questo risultato e che mi accolgono sempre a braccia aperte ad ogni mio rientro. Nominarvi tutti sarebbe impossibile.\\

\noindent Vorrei ringraziare Massimiliano, compagno di mille avventure universitarie. Sei stato il primo ragazzo conosciuto in triennale e da allora siamo stati sempre uno di fianco all'altro, uniti dalle stesse passioni, nell'affrontare le scelte relative alla specialistica e soprattutto gli innumerevoli progetti accademici. \\
Un ringraziamento speciale va ad Antonio, Francesco e Federico con i quali oltre a condividere questo fantastico percorso, è nata una splendida amicizia, che va al di fuori dell'ambito accademico. Grazie a loro e a Massi è stato possibile trascorrere fantastiche giornate, ma molto spesso anche pranzi, cene, etc. in giro per la bella Bologna.  \\

\noindent \texttt{Last but not least}, vorrei ringraziare due persone che ho conosciuto molto recentemente, ma si sono dimostrate sin da subito ottime compagne di viaggio. Seppur breve, per causa di forza maggiore, ma ha permesso, di esaltare, sin da subito, la vostra disponibilità e generosità sotto tutti i punti di vista. Vi ringrazio per aver reso meno pesanti, ma soprattutto molto più allegre quelle giornate in cui tutto era piuttosto grigio, a partire dalle ansie vissute e dallo studio, doveroso per superare il tanto temuto last exam, che in fondo è stata proprio la motivazione che ha sancito la nascita di questa stupenda amicizia. Grazie anche per le stupende giornate, seppur poche, passate assieme in quel di Bolo. Grazie Benedetta e grazie Chiara per essere state sempre al mio fianco, anche in un periodo difficile come questo e per aver sopportato e cercato di affievolire le mie ansie. 
Un immenso grazie a Benedetta per esserti dimostrata sin da subito una persona fantastica, con cui confidarsi e poter trascorrere stupende conversazioni, ma soprattutto voglio esprimere un caloroso ringraziamento per il prezioso tempo che hai dedicato nello scovare i refusi da me commessi. 
Grazie Chiara per avere reso non solo le giornate molto più allegre ma anche per aver reso il rientro verso i nostri alloggi molto più divertenti e meno duri, poiché guidati da splendide conversazioni.
Purtroppo il tempo trascorso assieme è stato davvero poco a causa di questo \texttt{COVID-19}, ma avremo modo e tempo per rifarci a conclusione di tutto ciò. Nel ringraziarvi, vi auguro anche un forte in bocca al lupo per il superamento di questo ostico ostacolo che, in fondo, ha sancito la nascita di questa brillante amicizia.\\

\noindent Ringrazio tutte le persone che non ho nominato esplicitamente in questa pagina, ma che hanno avuto un ruolo importante nella mia vita, perché i ricordi di tutti voi sono impressi in maniera indelebile nel mio cuore.